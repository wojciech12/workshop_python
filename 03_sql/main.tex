\documentclass[12pt, letterpaper]{article}
\usepackage[utf8]{inputenc}
\usepackage{hyperref}
\usepackage{enumerate}
\usepackage{listings}
\usepackage{fancyvrb}
\usepackage[T1]{fontenc}
\usepackage[document]{ragged2e} % keep all left

\usepackage{pmboxdraw} %verbatim utf8

\usepackage{minted} % yaml syntax highlighting

\newenvironment{markdown}%
    {\VerbatimEnvironment\begin{VerbatimOut}{tmp.markdown}}%
    {\end{VerbatimOut}%
        \immediate\write18{pandoc tmp.markdown -t latex -o tmp.tex}%
        \input{tmp.tex}}

\providecommand{\tightlist}{%
  \setlength{\itemsep}{0pt}\setlength{\parskip}{0pt}}

\newcommand*{\email}[1]{\href{mailto:#1}{\nolinkurl{#1}} }

\title{{\Huge \textbf{Wprowadzenie do {SQLa}}}}
%\author{Wojciech Barczynski}
\date{}


\begin{document}

\begin{titlepage}
\maketitle
\end{titlepage}

\tableofcontents
\pagebreak
\section{Uruchomienie bazy danych mysql}

Będziemy dzisiaj pracować na bazie danych mysql.

\subsection{Weryfikacja konfiguracji}

Sprawdźmy czy klient mamy właściwie działającego dockera (co to jest \emph{docker}
powiemy więcej na zajęciach o {\small CI}/{\small CD}):

\begin{minted}{bash}
$ sudo docker ps
\end{minted}

\subsection{Start}
Uruchommy bazę danych, przeklej poniższą komendę:

\begin{minted}{bash}
$ sudo docker run --name wsb-mysql \
   -e MYSQL_ROOT_PASSWORD=nomoresecret \
   -p 3306:3306 \
   -d mysql:8

# zweryfikuj czy dziala?
$ sudo docker ps
\end{minted}

Zauważ:
\begin{itemize}
\item zatrzymujemy bazę danych: \mintinline{bash}{docker stop wsb-mysql}
\item startujemy: \mintinline{bash}{docker start wsb-mysql}.
\end{itemize}

\subsection{Nawiązanie połączenia - {\small CLI}}
W pierwszej kolejności poznajmy klienta linii komend. Pierwszym krokiem jest
instalacja odpowiedniego pakietu:

\begin{minted}{bash}
$ sudo apt-get update
$ sudo apt install -y mysql-client
\end{minted}

Teraz nawiążmy połączenie:

\begin{minted}{bash}
# podaj hasło: nomoresecret
$  mysql -u root -h 127.0.0.1 -p
\end{minted}

I uruchommy pierwszą komendę mysqla: \mintinline{sql}{show databases;}:

\begin{minted}{text}
mysql> show databases;

+--------------------+
| Database           |
+--------------------+
| information_schema |
| mysql              |
| performance_schema |
| sys                |
+--------------------+
4 rows in set (0.00 sec)
\end{minted}

\subsection{Nawiązanie połączenia - {\small GUI}}
W dniu dzisiejszym skorzystamy z mysql-workbench\footnote{Dobrym wyborem byłby rówież \href{https://sequelpro.com/download}{sequelpro.com}}:

\begin{minted}{bash}
$ sudo apt update
$ sudo apt install mysql-workbench
\end{minted}

Po kilku minutach instalacji:

\begin{minted}{bash}
$ mysql-workbench &
\end{minted}

Teraz kliknij na \emph{Local Instance 3306} (użytkownik \emph{root}). Po nawiązaniu połączenia wykonaj po kolei następujące komendy:


\begin{itemize}
\item \mintinline{sql}{show databases;}
\item \mintinline{sql}{use mysql;}
\item \mintinline{sql}{show tables;}
\item \mintinline{sql}{select * from user;}
\end{itemize}

Jacy użytkownicy są utworzeni w naszej bazie?
\bigskip
\bigskip
\bigskip

Zauważ: praca na użytkowniku \textbf{root} jest niebezpieczna. Powinienes utworzyć osobnego użytkownika z dostępem tylko do baz danych wymaganych do pracy Twojej czy aplikacji.

%%%%%%%%%%%%%%%%%%%%%%
\section{Utworzenie bazy danych}

Skorzystamy z bazy danych używanych przez w3schools \footnote{Dobry tutorial znajdziesz również na \href{https://www.w3schools.com/sql/default.asp}{w3schools.com/sql}}.

\bigskip

1. Sciągnij \emph{w3schools.sql} z \emph{https://github.com/AndrejPHP/w3schools-database} (wyświetl zawartość pliku, potem wybierz Raw) używając \emph{wget}:

\begin{minted}{bash}
$ wget https://.. długi URL
$ ls

w3schools.sql}
\end{minted}

\bigskip 
2. Załaduj bazę danych:

\begin{minted}{bash}
# podaj hasło: nomoresecret
$  mysql -u root -h 127.0.0.1 -p

mysql> 
mysql> source w3schools.sql;

# zobaczmy czy wszystko OK
mysql> use w3schools;
mysql> show tables;

mysql> SELECT CustomerName,City FROM customers;
\end{minted}

Zauważ rezultat będzie podobny do\\ 
\href{https://www.w3schools.com/sql/trysql.asp?filename=trysql_select_columns}{https://www.w3schools.com/sql/trysql.asp?filename=trysql\_select\_columns} .


%%%%%%%%%%%%%%%%%%%%%%
\section{Ćwiczenia}
Według wskazówek wykładowcy.

\begin{enumerate}
 \item \verb|select|;
 \item \verb|select| i \verb|where|;
 \item \verb|count|;
 \item \verb|select| i \verb|join|;
 \item \verb|group| i \verb|order|;
 \item \verb|delete| i \verb|insert|.
\end{enumerate}

Notatki:
\bigskip
\bigskip
\bigskip
\bigskip
\bigskip

% Jeśli nie, są 3 opcje:

% \begin{itemize}
% \item modyfikujemy bezpośrednio w pliku \mintinline{bash}{$HOME/.gitconfig},
% \item dodajemy konfigurację \mintinline{bash}{.git/config} per repozytorium,
% \item Używamy komend \mintinline{bash}{git config} jak poniżej.
% \end{itemize}

% \begin{minted}{bash}

% # sprawdz konfiguracje (male L):
% $ git config -l
% $ git config --global user.name "wojciech11"

% # nie lubimy spamerow:
% $ git config --global user.email "wojciech11@users.noreply.github.com"
% \end{minted}

% \subsection{Przykładowe Repozytorium}

% Przygotujmy repozytorium na którym będziemy wykonywać ćwiczenia.

% \bigskip
% 0. Utwórz folder o nazwie takiej jak nazwa Twojego użytkownika GIT.

% \bigskip
% 1. Utwórz repozytorium o nazwie \mintinline{text}{nauka_gita_2}

% \bigskip
% 2. Utwórz plik README.md i dodaj go do repozytorium git:

% \smallskip
% \begin{minted}[frame=single]{text}
% # Testy platforma automatyzacji procesów finansowych z ML

% ## Spec

% ## Referencje

% \end{minted}

% \bigskip
% 2. Wypisz na ekran zawartość \mintinline{bash}{.git/config}

% \bigskip
% 3. Utwórz repozytorium na githubie i ustaw \textit{git remote} dla naszego repozytorium.

% \bigskip
% 4. Co się zmieniło w \mintinline{bash}{.git/config}?

% \bigskip
% 5. Korzystając z \href{https://guides.github.com/features/mastering-markdown/}{https://guides.github.com/features/mastering-markdown/} zmodyfikuj plik README dwukrotknie. Za każdym razem wykonaj $push$ do githuba.

% \pagebreak
% \section{git branch}

% Branche służą do pracy na różnych wersjach naszego kodu równolegle, najlepszą praktyką są krótko żyjące branche. Najczęstsze przypadki użycia:
% \begin{itemize}%
% \item praca równoległa nad jednym projektem,%
% \item ekspertymenty,%
% \item długo żyjące branche na oznaczenie środowisk deweloperskich: $prod$ i $dev$.%
% \end{itemize}%


% \bigskip
% 1. Utwórz branch \mintinline{text}{add_TODO_to_README}

% \begin{minted}{bash}
% $ git checkout -b add_TODO_to_README
% # zauważ:
% $ git branch
% $ git branch --remote
% \end{minted}

% $git\ branch$ pozwala nam szybko sprawdzić gdzie jesteśmy.

% \bigskip
% 2. Komendą do zmiany branch to też $checkout$. Wykonaj następujące komendy:

% \begin{minted}{bash}
% $ git checkout master
% $ git branch
% $ git checkout add_TODO_to_README
% $ git branch
% \end{minted}

% \bigskip
% 3. Na nowym branchu, w trzech commitach dodaj tekst do sekcji \mintinline{text}{TODO}.

% \bigskip
% 4. Wykorzystując $gitk$ i $gitg$, zobacz jak wygląda nasze repozytorium.

% \bigskip
% 5. Korzystając z komendy $checkout$, wyświetl zawartość pliku README na branchu $master$.

% \bigskip
% 6. Wróć na branch \mintinline{text}{add_TODO_to_README} i umieść zmiany na githubie za pomocą komendy \mintinline{bash}{git push}.

% \bigskip
% 7. Zweryfikuj, że możesz zobaczyć nowy branch na githubie

% \bigskip
% 8. Jeśli jesteś zadowolony ze stanu nowej sekcji, umieść zmiany na $master$ z pomocą komendy \mintinline{bash}{git merge}:

% \begin{minted}{bash}
% # wracamy na branch master
% $ git checkout master
% $ git branch
% # domyslnie merge-ujemy zmiany na branch
% # na ktorym jestesmy
% $ git merge add_TODO_to_README
% $ git push
% \end{minted}

% \section{Pull requests na githubie}
% Praca oparta na Pull Requestach \verb|PR| jest standardem zarówno na githubie, jak i gitlabie (\href{https://gitlab.com}{gitlab.com}) czy bitbuckecie (\href{https://bitbucket.com}{bitbucket.com}).

% \bigskip

% 1. Utwórz branch \mintinline{text}{add_SMOKE_TEST_to_README}, dodaj sekcję \mintinline{text}{## Smoke tests}, w dwóch commitach dodaj treść do nowej sekcji.

% \bigskip
% 2. Zweryfikuj:

% \begin{minted}{bash}
% # czy wszystko dodalismy?
% $ git status
% # czy mamy 2 commity
% $ git log
% # oraz graficznie
% $ gitk
% \end{minted}

% \bigskip
% 3. Umieść brancha \mintinline{text}{add_SMOKE_TEST_to_README} na githubie.

% \bigskip
% 4. Poczekaj na prowadzącego. Miejsce na notatki:
% \bigskip
% \bigskip
% \bigskip

% 5. Zaktualizujmy nasze lokalne repozytorium:

% \begin{minted}{bash}
% $ git pull
% # jak jest roznica:
% $ git pull --rebase
% \end{minted}

% %%
% \pagebreak
% \section{Cofnięcie ostatnich zmian}

% Po dodaniu nowej sekcji "Dokumentacja", dochodzimy do wniosku, że jedenak nie jest ona nam potrzebna. Z pomocą komendy \mintinline{bash}{git checkout} możemy anulować zmiany.

% \bigskip
% 1. Dodaj sekcję "Dokumentacja"

% \bigskip
% 2. Sprawdź status pliku README, powinien być \textit{modified}.

% \bigskip
% 3. Analujmy ostatnie zmiany:

% \begin{minted}{bash}
% # Czy modified?
% $ git status
% # wrocmy do niezmodyfikowanej wersji
% $ git checkout README.md
% \end{minted}

% Zauważ: mamy również komendę \mintinline{bash}{git stash}, gdzie możemy odłożyć zmiany do późniejszej pracy poza repozytorium gita.

% %%
% \section{Confnięcie się do wersji}

% Ćwiczenie z prowadzącym.

% %%
% \section{Git tag}

% \begin{minted}{bash}
% $ git tag -a v1.0.0 -m "Testy dla wersji 1.0.0 platformy"
% # pojedynczy tag:
% git push tag v1.0.0
% # wszystkie tagi
% $ git push --tags
% \end{minted}

% %%
% \pagebreak
% \section{Rozwiązywanie konfliktów}

% W tym ćwiczeniu zobaczymy jak pracować z konfliktami w repozytorium git. Proszę pamiętać, że najlepszą metodą na unikanie konfliktów są krótko żyjące branche.

% \bigskip
% 1. Utwórzmy branch \mintinline{text}{add_EXTERN_DOCS_to_README}, dodaj sekcję \mintinline{text}{## Extern Docs}, commit i push.

% \bigskip
% 2. Dodaj parę linii kodu z poziomu edytowa na stronie github.com do nowej sekcji w naszym pliku README i zapisz.

% \bigskip
% 3. Lokalnie, również zmodyfikuj sekcję, zrób commit.

% \bigskip
% 4. Wykonaj \mintinline{text}{git pull}, co się stanie?

% \bigskip
% 5. Za wskazówkami wykładowcy, rozwiąż konflikt, oraz ostateczne umieść na githubie.

% \bigskip
% 6. Używając PR albo \mintinline{text}{git merge} umieść zmiany na branchu \mintinline{text}{master}.

% %%
% \section{Git workflows}

% Znajdź używając google-a, czym charakteryzują się następujące konwencje użycia gita:

% \begin{itemize}%
% \item GitFlow%
% \bigskip
% \bigskip
% \bigskip

% \item Github Flow%
% \bigskip
% \bigskip
% \bigskip
% \end{itemize}%

% Zastanów się i odpowiedź na pytania:

% \begin{itemize}%

% \item Jaki wpływ workflow ma na sposób budowy oprogramowania?
% \bigskip
% \bigskip
% \bigskip

% \item Jakie są zalety i wady GitFlow?
% \bigskip
% \bigskip
% \bigskip

% \item Jakie są zalety i wady Github Flow?
% \bigskip
% \bigskip
% \bigskip

% \item Ciekawostka. Jak wygląda rozwój Linux kernel?
% \bigskip
% \bigskip
% \bigskip

% \end{itemize}%

% \section{Podsumowanie}

% Na samym początku, jeśli mamy jakiekolwiek wątpliwości co robimy:

% \begin{itemize}%
% \item Skopiuj zawartość całego folderu jako backup.
% \item Przed większymi operacjami na danych branchu utwórz dodatkowy.
% \item Nie bój się używać środowiska graficznego: gitk, gitg czy sourcetree.
% \item Unikaj \mintinline{text}{--force}
% \end{itemize}%

\begin{markdown}
\end{markdown}

\end{document}
